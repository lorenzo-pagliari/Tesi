\chapter{An introduction to the writing of scientific texts}
\label{chap:aChapter}
\begin{flushright}{\slshape    
   Science, my boy, is made up of mistakes, but they are mistakes
   which it is useful to make, because they lead little by little
   to the truth}. \\ \medskip --- \citeauthor{verne_journey:1957}
   \citetitle{verne_journey:1957}
\end{flushright} 

\section{The structure of a scientific text}

\section{Bibliographies and literature reviews}

\section{A tentative index}

\section{Follow the instructions}
Visit \href{http://www.tedoc.polimi.it/tesilaurea/Consegna-tesi-di-laurea-(vecchio-ordinamento-e-specialistica)}{this link} for the updated information about the content of the thesis.

\enquote{Alcune Scuole forniscono linee guida specifiche cui i laureandi devono attenersi per la redazione della tesi. Per ulteriori informazioni:
\href{http://www.tedoc.polimi.it/download/lauree_magistrali/201406_POLITesi_Info_specifiche_scuole.pdf}{www.tedoc.polimi.it/\ldots}}

\subsection{Archiving electronic documents: PDF/A}
PDF/A is an ISO-standardized version of the Portable Document Format (PDF) specialized for the digital preservation of electronic documents. PDF/A differs from PDF by prohibiting features ill-suited to long-term archiving, such as font linking (as opposed to font embedding). The ISO requirements for PDF/A file viewers include color management guidelines, support for embedded fonts, and a user interface for reading embedded annotations.

Universities usually requires this standard but they're also not aware that common programs like MS Word, OpenOffice and so on aren't really able to produce compliant PDFs. In Latex, there's some development going on but at the time of writing, the available commands are still too obscure and buggy. So in the end, forget the PDF/A for now.\footnote{Or DIY and then make a pull request on github :D.}