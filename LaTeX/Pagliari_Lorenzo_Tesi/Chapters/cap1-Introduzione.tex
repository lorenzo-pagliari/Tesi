\chapter{Introduzione}
\label{chap:Introduzione}

Il lavoro che presentiamo si colloca nell'ambito dello studio sul consumo energetico, nell'ambito della comunicazione tra dispositivi mobili e nell'ambito della diffusione di informazioni. Abbiamo affrontato queste tre tematiche nell'analisi di particolari. Può accadere che, sotto particolari condizioni o eventi, le comuni reti di comunicazione, quali reti telefoniche fisse, mobili e rete Internet, possano non essere disponibili per un certo lasso di tempo. Non stiamo parlando di disservizi dovuti a interventi di ordinaria manutenzione, ma di mancato servizio causato da eventi esterni che possono intaccare queste infrastrutture. Le più comuni cause di questo tipo sono i fenomeni atmosferici. Fenomeni meteorologici di forte intensità e potenza, come nubifragi, tempeste, trombe d'aria, forti nevicate, alluvioni e così molte altri, possono danneggiare le infrastrutture di comunicazione causando la mancanza dei servizi. Infatti è comune che dopo forti nevicate, grandi piogge o alluvioni, vengano a mancare i servizi telefonici e non si riesca a comunicare. In tali situazioni si resta senza reti di comunicazione anche per giorni senza la possibilità di avere notizie di alcun tipo, anche dagli organi competenti addetti a intervenire. Situazioni di questo tipo non sono le sole, ci può essere anche la mancanza volontaria dei servizi telefonici. Questo tipo di comportamento è stato e viene tuttora usato contro cortei di protesta, col tentativo di togliere alle persone il principale mezzo di organizzazione della protesta \cite{wemakehistory2014-articolo}, \cite{wemakehistory2014-fattoq}, \cite{wemakehistory2014-lastampa}. Il nostro studio ha afforntato questo problema cercando di proporre una soluzione che possa essere utilizzata in queste situazioni di emergenza. Come sottolinea l'indagine ISTAT di dicembre 2014\cite{istat2014}, la tecnologia ormai è qualcosa di molto comune tra persone, quindi abbiamo cercato di analizzare in che modo fosse possibile sfruttare per diffondere le informazioni cio' che normalmente le persone hanno con se, come telefoni cellulari o tablet, ipotizzando che ne siano in possesso anche in queste situazioni di emergenza. In questo caso si puo' immaginare, da un punto di vista di rete di comunicazione, di avere un insieme di nodi che posso solo vedere e comunicare solo con i nodi a loro vicini. Questo si traduce in una rete Peer-to-Peer, ovvero una rete senza \textit{overlay}, in cui non vi sono strutture gerarchiche e i nodi, chiamati \textit{peero}, comunicano tra loro solo in modalità punto-punto (da cui \textit{Peer-to-Peer}), senza alcuna struttura di tipo Clinet-Server. Presenteremo anche la tecnologia di comunicazione molto diffusa ed equipaggiata su tutti i dispositivi mobili da molti anni: il Bluetooth. La nostra idea principale in fase di progettazione della soluzione, è stata quella di simulare il comportamento umano in queste situazioni: fare il passa parola. Questo tipo di comportamento rappresenta un ottimo modo di diffondere informazioni tra le persone se non vi sono altri mezzi ed è uguale al gossip. Quando le persone vengono a conoscenza di uno scoop, un'informazione nuova, cominciano a comunicarlo ad altre persone, che a loro volta lo diranno ad altre ancora e così via. Questo è esattamente il comportamento definito come "gossip". Questo concetto vale anche per i social network, dove una nuova notizia viene \textit{tweetata} o condivisa molte volte per cercare di diffonderla il più possibile. Per questa ragione abbiamo scelto di studiare e utilizzare gli algoritmi di gossip, chiamati anche algoritmi epidemici. In origine questi algoritmi vennero creati per modellare il comportamento epidemico delle malattie, ma poi si scoprì che si adattano molto bene nel modellare anche il comportamento di gossip. Presenteremo alcuni dei più comuni algoritmi di gossip e come abbiamo utilizzato questi concetti nel progettare la nostra soluzione. Parleremo anche di software e più precisamente di simulatori di reti e protocolli. Presenteremo quelli più diffusi presenti in letteratura, le loro caratteristiche e le loro funzionalità. Mostreremo anche quale simulatore è stato utilizzato per il lavoro e come esso è stato impostato.

Dalla scelta dei dispositivi da utilizzare, quali ad esempio gli smartphone, è sorto anche un altro aspetto da dover analizzare: quello del consumo energetico. E' risaputo che l'autonomia è uno dei punti deboli di questi dispositivi, quindi abbiamo lavorato per gestire questo aspetto. Presenteremo lo studio di fattibilità energetica fatto sulla tecnologia Bluetooth presa in considerazione, poi discuteremo di come questo problema del consumo energetico è stato gestito.

La soluzione che proponiamo in questo lavoro quindi è un sistema che cerca di diffondere informazioni, cercando di replicare il passa parola fatto dalle persone. Questo sistema si basa sulla tecnologia di trasmissione Bluetooth, presente su ogni dispositivo mobile moderno o di ultima generazione, e la diffusione dell'informazione è guidata dai principi degli algoritmi di gossip; il sistema inoltre gestisce sempre il consumo energetico calibrando il carico di lavoro dinamicamente, in base alle condizioni esterne e interne del dispositivo. Infine presenteremo i risultati ottenuti e le nostre considerazioni.

\section{Struttura del documento}
In questo documento presentiamo il lavoro svolto, organizzato con la seguente struttura:
\begin{itemize}
	\item \textit{Cap.2-Stato dell'arte}: in questo capitolo presentiamo lo stato dell'arte concernente le tecnologie, i modelli studiati, gli strumenti utilizzati e gli studi su cui questo lavoro si basa. Discuteremo degli studi sul risparmio energetico, di come questo aspetto sia sempre di maggiore importanza e degli studi fatti fin ora. Presenteremo le caratteristiche della più recente tecnologia di Bluetooth, elencando i principali aspetti studiati e utilizzati in questo lavoro. Illustreremo quell'insieme di modelli di rete adatti alla rappresentazione di reti Peer-to-Peer. Successivamente introdurremo il principio di Gossip e presenteremo i principali algoritmi epidemici presenti in letteratura ed infine discuteremo un insieme di strumenti di simulazione e cercheremo di approfondire gli aspetti di due di essi.
	\item \textit{Cap.3-Introduzione del problema}: in questo capitolo, discuteremo come il problema è stato affrontato e come abbiamo impostato il processo di studio. Presenteremo il caso di studio relativo al consumo energetico, che sta alla base della lavoro svolto.
	\item \textit{Cap.4-Progettazione logica}: in questo capitolo presentiamo la fase di progettazione logica. Illustreremo come abbiamo strutturato il lavoro e le scelte implementative fatte per i vari moduli del sistema.
	\item \textit{Cap.5-Architettura del sistema}: in questo capitolo presentiamo l'architettura del sistema, in un ottica implementativa. Illustreremo come è stato implementato tutto ciò che è stato presentato nel Capitolo 4.
	\item \textit{Cap.6-Simulazioni e valutazione risultati}: in questo capitolo discuteremo inizialmente la parte di realizzazione, simulazione e raccolta dati per poi presentare, valutare e discutere i dati raccolti.
	\item \textit{Cap.7-Conclusioni e sviluppi futuri}: in questo capitolo parleremo delle possibili direzioni future da poter intraprendere per proseguire questo lavoro. Discuteremo delle ipotesi fatte in questo lavoro e di altri aspetti da poter studiare e analizzare per proseguire e migliorare il lavoro. Presenteremo infine le nostre conclusioni in merito al lavoro svolto.
\end{itemize}