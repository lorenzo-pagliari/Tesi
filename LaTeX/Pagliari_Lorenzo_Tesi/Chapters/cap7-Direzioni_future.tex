\chapter{Direzioni Future}
\label{chap:direzioni_future}

Direzioni e studi futuri si possono concentrare sul considerare la mobilità dei nodi, in quando persone; dispositivi contagiati che si spostano possono incrementare le prestazioni in termini di copertura della rete, proprio come nei casi delle epidemie. Studiare se vi sono possibilità di sfruttare maggiormente tutta la piconet e quindi poter associare più dispositivi slave per ogni master. Studiare migliori modelli di reti e migliori metodi per simulare la distribuzione abitativa dei paesi; in ogni città, gli abitanti non sono mai distribuiti uniformemente su tutta l'area sotto la giurisdizione comunale. Lo studio di eventuali pattern o di modelli più accurati, può incrementare l'efficienza anche a densità basse. Lo studio di una maggiore gestione dei messaggi, magari con l'inserimento di un TTL nei messaggi e politiche di rinvio di messaggi che erano già stati accantonati, al rilevamento di nuovi dispositivi nell'area circostante.