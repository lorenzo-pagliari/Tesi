\chapter{Conclusioni e sviluppi futuri}
\label{chap:conclusioni_sviluppiFuturi}

La nostra ricerca si pone nell'ambito dello studio per la riduzione e ottimizzazione del consumo energetico e nell'ambito della comunicazione tra dispositivi mobili.  Il nostro lavoro si è concentrato sullo studio di una possibile soluzione alla mancanza delle comuni reti di comunicazioni, quali reti telefoniche e internet. Come risultato del nostro lavoro abbiamo definito un algoritmo (adattativo) progettato come estensione di un algoritmo di gossip, che sfrutta come canale di trasmissione la tecnologia Bluetooth Low Energy, presente su tutti i più comuni dispositivi mobili in commercio. Il nostro algoritmo sfrutta le caratteristiche del gossip per diffondere informazioni e grazie alla sua capacità di adattamento, cerca sempre di trovare un compromesso nel definire il carico di lavoro del dispositivo tra autonomia ed efficienza. Questo per garantire un buon consumo energetico, e non degradare troppo l'autonomia del dispositivo, e allo stesso tempo offrire un'efficiente azione di gossip. Abbiamo analizzato il sistema per diverse situazioni che tengono conto sia delle densità che della dimensione della popolazione coinvolta.

Possibili sviluppi di questo lavoro consistono nell'includere la mobilità dei nodi, in quando persone; dispositivi contagiati che si spostano possono incrementare le prestazioni in termini di copertura della rete, proprio come nei casi delle epidemie. Un'altra possibile direzione di ricerca consiste nello studio di possibili modi per sfruttare maggiormente tutta la piconet e quindi poter associare più dispositivi slave per ogni master. Altra possibilità di estensione è lo studio di modelli di reti migliori e metodi più efficienti per simulare la distribuzione abitativa dei paesi; in ogni città, gli abitanti non sono mai distribuiti uniformemente su tutta l'area sotto la giurisdizione comunale. Un'altra possibile direzione è lo studio di eventuali pattern o di modelli più accurati, può incrementare l'efficienza anche a densità basse. Un'altra direzione è anche lo studio per una miglior gestione dei messaggi, magari con l'inserimento di un \acf{TTL} e politiche di rinvio di messaggi che erano già stati accantonati, al rilevamento di nuovi dispositivi nell'area circostante.