\chapter{Introduzione}
\label{chap:Introduzione}

\section{The structure of a scientific text}

ciao come va ??\\
primo ref \cite{ranganathan2010-pac}
\ac{BT} .. \acs{BLE} ... \acp{BLE}... \acf{DF} ... \acl{FF}

\subsection{Archiving electronic documents: PDF/A}
PDF/A is an ISO-standardized version of the Portable Document Format (PDF) specialized for the digital preservation of electronic documents. PDF/A differs from PDF by prohibiting features ill-suited to long-term archiving, such as font linking (as opposed to font embedding). The ISO requirements for PDF/A file viewers include color management guidelines, support for embedded fonts, and a user interface for reading embedded annotations.



Universities usually requires this standard but they're also not aware that common programs like MS Word, OpenOffice and so on aren't really able to produce compliant PDFs. In Latex, there's some development going on but at the time of writing, the available commands are still too obscure and buggy. So in the end, forget the PDF/A for now.\footnote{Or DIY and then make a pull request on github :D.}

\subsection{Struttura del documento}
Questo documento è stato strutturato così:
\begin{itemize}
	\item Capitolo 1: Introduzione
	\item Capitolo 2: Stato dell'arte
	\item Capitolo 3: Introduzione del problema
	\item Capitolo 4: Progettazione logica
	\item Capitolo 5: Architettura del sistema
	\item Capitolo 6: Simulazioni e 
	\item Capitolo 7: Direzioni future e conclusioni
\end{itemize}