\chapter{Conclusioni}
\label{chap:conclusioni}

La nostra ricerca si pone nell'ambito dello studio per la riduzione e ottimizzazione del consumo energetico e nell'ambito dei dispositivi mobile.  Il nostro lavoro si è concentrato per lo studio e la ricerca di una possibile soluzione alla mancanza delle comuni reti di comunicazioni, quali reti telefoniche e internet. La nostra ricerca ha prodotto un algoritmo dinamico, progettato come estensione di un algoritmo di gossip, che sfrutta come canale di trasmissione la tecnologia Bluetooth Low Energy, equipaggiata su tutti i più comuni dispositivi mobili in commercio. Il nostro algoritmo sfrutta le caratteristiche del gossip per diffondere informazioni ma grazie al suo dinamismo, cerca sempre di trovare un compromesso nello scegliere il carico di lavoro del dispositivo, per garantire un buon consumo energetico per non degradare troppo l'autonomia del dispositivo ma allo stesso tempo un'efficiente azione di gossip. Abbiamo analizzato il sistema per varie densità e per due diversi potenziali raggi d'azione. I risultati hanno mostrato un'ottima applicabilità fino a densità $d=0.001\, nodi/m^2$, con una copertura nell'intorno del 90\%, mentre per densità più piccole un lineare degrado delle prestazioni all'aumentare del numero di nodi nella rete, a causa della formazione di sottoreti isolate. In prima analisi, i risultati sono promettenti sia nell''ottica di futuri studi sia nell'ottica di un miglioramento delle prestazioni della tecnologia Bluetooth.